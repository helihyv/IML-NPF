\documentclass[a4size, 12pt]{report}
\begin{document}
	
	\author{Group 19 – Heli Hyvättinen}
	\title{Introduction to Machine Learning – Term project initial report }
	
	The research question
	
	The subject of study is the phenomenom of new particle formation (NPF). In new particle formation small particles start form larger particles. Our classifier models under what consdtions NPF happens. It is the used to predict first whether NPF happens and then additionally what kind of NPF happens. Also the propabilities of these events and lack of events are reported for both the binary- and 4 class classifier.  
	     
	
	The Data
	
	The data consist of daily means and standard deviations of measurements of several conditions and the occurrence of NPF evens at the Hyytiälä forestry field station.There are three kinds of NPF events in the data, labeled "Ia", "Ib" and "II". Labels "Ia" and "Ib" are used for the days om which growth and formation date was determined with a good confidence level (Source Maso et al. 2005). Other NPF events are lab2led "II".
	n  Class "Ia" means that a clear and strong NPF event happened on that day, while the label "Ib" is used for other days that fulfill the conditions for label "I")  The fourth possibility for given day is that no NPF event happened at all. The data is guaranteed not to have missing values. The data has been reported to have an equal amount of event ands nonevent days despite the nonevent days being more frequent in real life.
	
	The Methods
		
	-describe approaches considered
	-approach chosen
	-apros and cons of this approach
	
	Binary Classification
	
		Results
	
	Data Exploration
	
	Feature Selection
	
	Model Selection
	
	Model Created
	
	Predictions
	
	Classification accuracy Estimation
	
	Perplexity
	
	Multi-label Classification
	
	
	Results
	
	Data Exploration
	
	Feature Selection
	
	Model Selection
	
	Model Created
	
	Predictions
	
	Classification accuracy Estimation
	
	
%	Self-grading report
	     

\end{document}